
\newcounter{qcntvalue}
\setcounter{qcntvalue}{5}

% Generate the table header and the empty answer row
\newcommand{\tableheader}{}
\newcommand{\answerrow}{}
\foreach \n in {1,...,\value{qcntvalue}}{
		\ifnum\n=\value{qcntvalue}
			\xdef\tableheader{\tableheader Q\n} % No '&' for the last item
			\xdef\answerrow{\answerrow\strut} % No '&' for the last item in the answer row
		\else
			\xdef\tableheader{\tableheader Q\n &} % '&' for items except for the last
			\xdef\answerrow{\answerrow & \strut} % '&' for items except for the last in the answer row
		\fi
	}

\newcolumntype{C}[1]{>{\centering\arraybackslash}m{#1}}

\vspace{0.5cm}
{\small
	\begin{spacing}{1.2}
		\tcolorbox[colframe=black, colback=white, boxrule=1pt, width=\textwidth]
		% Main Table Content
		\renewcommand{\arraystretch}{1.4}
		\begin{center}
			\textbf{Couleur globale} \\[0.3cm]
			\begin{tikzpicture}
				% Define sizes for arrows
				\def\cellWidth{1.3} % Width of the rectangular part
				\def\cellHeight{1}  % Height of the arrow
				\def\tipWidth{0.5}  % Width of the arrow tip

				% Table entries
				\newcommand{\tableEntries}{R1,R2,R3,O1,O2,O3,V1,V2,V3,B1,B2,B3}

				% Initial position
				\coordinate (start) at (0,0);

				% Draw each arrow
				\foreach \entry [count=\i] in \tableEntries {
					% Define points of the arrow
					\coordinate (A) at (start);                               % Bottom-left
					\coordinate (B) at ($(A) + (\cellWidth,0)$);             % Bottom-right
					\coordinate (C) at ($(B) + (\tipWidth,0.5*\cellHeight)$); % Arrow tip
					\coordinate (D) at ($(B) + (0,\cellHeight)$);            % Top-right
					\coordinate (E) at ($(A) + (0,\cellHeight)$);            % Top-left
					\coordinate (F) at ($(A) + (\tipWidth,0.5*\cellHeight)$); % Top of tip

					% Draw arrow
					(A) -- (B) -- (C) -- (D) -- (E) -- (F) -- cycle; % Fill the arrow
					\draw[thick] (A) -- (B) -- (C) -- (D) -- (E) -- (F) -- cycle;   % Border for the arrow

					% Place text in the arrow
					\node at ($(A)!0.5!(C)+(0,0.25*\cellHeight)$) {\textbf{\entry}};

					% Update start position for next arrow (move to arrow tip start)
					\coordinate (start) at (B); % Shift only to the rectangle's right edge
				}
			\end{tikzpicture}
		\end{center}
		\vspace{0.3cm}
		\tcblower

		\vspace{0.3cm}
		\renewcommand{\arraystretch}{1.5}
		\begin{center}
			\textbf{Couleur par question} \\[0.3cm]
			\begin{tabular}{| *{12}{C{1.5cm}|} }
				\hline
				\tableheader \\ % Use the header constructed above
				\hline
				% \rule{0pt}{6ex}\answerrow \\ % Use the answer row constructed above
				% \hline
			\end{tabular}
		\end{center}
		\vspace{0.3cm}

		% \noindent\makebox[\textwidth]{\rule[0.5ex]{\textwidth}{0.5pt}}\\[0.3cm]

		\textbf{Remarques : } \\
		\versions{}{}{}{Version EABS.}

		\endtcolorbox
	\end{spacing}
}


\vspace{0.5cm}
{ \small
	\setstretch{1}
	\tcolorbox[colframe=black, colback=white, boxrule=1pt, width=\textwidth]
	\subsection*{Objectifs}

	% \noindent
	% \begin{tabularx}{\textwidth}{|X| *{2}{c|}}
	% 	\hline
	% 	% \textbf{Je suis capable de...}                                                                      & \textbf{1} & \textbf{2} & \textbf{3} & \textbf{4} & \textbf{5} \\
	% 	\textbf{Je suis capable de...}                                                                      &           &           \\
	% 	\hline
	% 	Effectuer des divisions Euclidienne par calcul écrit.                                               & \ding{51} & \ding{55} \\
	% 	\hline
	% 	Formuler et utiliser dans différents contextes la relation fondamentale de la division Euclidienne. & \ding{51} & \ding{55} \\
	% 	\hline
	% 	Reconnaitre des nombres premiers entre eux.                                                         & \ding{51} & \ding{55} \\
	% 	\hline
	% 	Déterminer l’ensemble des diviseurs et des multiples d’un nombre.                                   & \ding{51} & \ding{55} \\
	% 	\hline
	% 	Déterminer les diviseurs communs et le PGCD de deux nombres ou plus.                                & \ding{51} & \ding{55} \\
	% 	\hline
	% 	Déterminer les multiples communs et le PPCM de deux nombres ou plus.                                & \ding{51} & \ding{55} \\
	% 	\hline
	% 	Résoudre des problèmes relatifs au PGCD ou PPCM.                                                    & \ding{51} & \ding{55} \\
	% 	\hline
	% \end{tabularx}

	\textbf{Je suis capable de...}
	\begin{itemize}[label=\checkboxEmpty]
		\item Construire, à l’aide d’instruments, l’image d’une figure par une rotation.
		\item Déterminer des rotations dans des polygones réguliers et des figures infinies.
		\item Rechercher des points fixes et droites fixes pour chacune des transformations du plan.
		\item Construire l’image d’une droite et d’une demi-droite par chacune des transformations.
		\item Utiliser les propriétés des invariants pour effectuer des constructions rapides ou pour justifier des caractéristiques de figures.
	\end{itemize}

	\endtcolorbox
}
