
\newcounter{qcntvalue}
\setcounter{qcntvalue}{5}
\newcommand{\Total}{18} % total des points/pages
\newcommand{\HalfTotal}{\inteval{\Total/2}} % calcul automatique de la moitié

% Generate the table header and the empty answer row
\newcommand{\tableheader}{}  % Initialize empty header
\newcommand{\answerrow}{}    % Initialize empty answer row
\foreach \n in {1,...,\value{qcntvalue}}{
		\xdef\tableheader{\tableheader Q\n &} % Add '&' for all items
		\xdef\answerrow{\answerrow \strut &}  % Add '&' for all items
	}

% Add an extra empty cell at the end
\xdef\tableheader{\tableheader ~}  % Append an empty cell
\xdef\answerrow{\answerrow \strut} % Append an empty answer cell

\newcommand{\drawQcircles}{%
	\begin{center} % Center the TikZ picture on the page
		\begin{tikzpicture}[every node/.style={draw, thick, circle, minimum size=1cm, inner sep=2pt, fill=white}]
			% Define horizontal and vertical separation
			\def\circleSep{1.4}
			\def\verticalSep{0.5}
			% Starting point for the first pair of circles
			\coordinate (start) at (0,0);
			% Loop from 1 to the value of qcntvalue
			\foreach \n in {1,...,\value{qcntvalue}} {%
					% Bottom empty circle positioned vertically below the Q circle (also filled white)
					\node (A\n) at ($(start)+(0,-\verticalSep)$) {};

					% Top circle with Q label (filled white)
					\node (Q\n) at (start) {\textbf{Q\n}};

					% Move the starting point to the right for the next pair
					\coordinate (start) at ($(start)+(\circleSep,0)$);
				}
		\end{tikzpicture}
	\end{center}%
}


\newcolumntype{C}[1]{>{\centering\arraybackslash}m{#1}}

% \vspace{0.5cm}
\vfill
{\small
	\begin{spacing}{1.2}
		\tcolorbox[colframe=black, colback=white, boxrule=1pt, width=\textwidth]
		% Main Table Content
		\renewcommand{\arraystretch}{1.4}
		\begin{center}
			\textbf{Couleur globale} \\[0.3cm]
			\begin{tikzpicture}
				% Define sizes for arrows
				\def\cellWidth{1.3} % Width of the rectangular part
				\def\cellHeight{1}  % Height of the arrow
				\def\tipWidth{0.5}  % Width of the arrow tip

				% Table entries
				\newcommand{\tableEntries}{R1,R2,R3,O1,O2,O3,V1,V2,V3,B1,B2,B3}

				% Initial position
				\coordinate (start) at (0,0);

				% Draw each arrow
				\foreach \entry [count=\i] in \tableEntries {
					% Define points of the arrow
					\coordinate (A) at (start);                               % Bottom-left
					\coordinate (B) at ($(A) + (\cellWidth,0)$);             % Bottom-right
					\coordinate (C) at ($(B) + (\tipWidth,0.5*\cellHeight)$); % Arrow tip
					\coordinate (D) at ($(B) + (0,\cellHeight)$);            % Top-right
					\coordinate (E) at ($(A) + (0,\cellHeight)$);            % Top-left
					\coordinate (F) at ($(A) + (\tipWidth,0.5*\cellHeight)$); % Top of tip

					% Draw arrow
					(A) -- (B) -- (C) -- (D) -- (E) -- (F) -- cycle; % Fill the arrow
					\draw[thick] (A) -- (B) -- (C) -- (D) -- (E) -- (F) -- cycle;   % Border for the arrow

					% Place text in the arrow
					\node at ($(A)!0.5!(C)+(0,0.25*\cellHeight)$) {\textbf{\entry}};

					% Update start position for next arrow (move to arrow tip start)
					\coordinate (start) at (B); % Shift only to the rectangle's right edge
				}
			\end{tikzpicture}

			{\small
			Total = \ddotfill[1cm]/\Total\\
			\textit{(V1 = \HalfTotal/\Total)}
			}
		\end{center}
		% \vspace{0.3cm}

		% ---------------------------------------------------------
		\tcblower

		\vspace{0.3cm}
		\begin{center}
			\textbf{Couleur par question} \\[0.3cm]
			% 	\begin{tabular}{| *{12}{C{1.2cm}|} }
			% 		\hline
			% 		\tableheader \\ % Use the header constructed above
			% 		\hline
			% 		\answerrow   \\    % Expands to "& \strut & \strut ... & \strut"
			% 		\hline
			% 		% \rule{0pt}{6ex}\answerrow \\ % Use the answer row constructed above
			% 		% \hline
			% 	\end{tabular}
			\drawQcircles
		\end{center}
		\vspace{0.3cm}

		% ---------------------------------------------------------
		\tcbline

		\textbf{Remarques : } \\
		\versions{}{}{}{Version EABS.}

		\endtcolorbox
	\end{spacing}
}

% \vspace{0.5cm}
\vfill
{ \small
	\setstretch{1}
	\tcolorbox[colframe=black, colback=white, boxrule=1pt, width=\textwidth]
	\subsection*{Objectifs}

	\textbf{Je suis capable de...}
	\begin{itemize}[label={\bfseries \checkboxEmpty}]
		\item Comparer les croissances des fonctions exponentielles.
		\item Résoudre une équation exponentielle simple.
		\item Utiliser une fonction logarithme ou exponentielle pour résoudre un problème.
	\end{itemize}

	\endtcolorbox
}
\vfill
