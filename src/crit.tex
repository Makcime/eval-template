
\newcounter{qcntvalue}
\setcounter{qcntvalue}{5}

% Generate the table header and the empty answer row
\newcommand{\tableheader}{}
\newcommand{\answerrow}{}
\foreach \n in {1,...,\value{qcntvalue}}{
		\ifnum\n=\value{qcntvalue}
			\xdef\tableheader{\tableheader Q\n} % No '&' for the last item
			\xdef\answerrow{\answerrow\strut} % No '&' for the last item in the answer row
		\else
			\xdef\tableheader{\tableheader Q\n &} % '&' for items except for the last
			\xdef\answerrow{\answerrow & \strut} % '&' for items except for the last in the answer row
		\fi
	}
\newcolumntype{C}[1]{>{\centering\arraybackslash}m{#1}}


\section*{}
 % Targeted competences and table for grading can be set here. For example:
 { \small
  \begin{spacing}{1.2}
	  \begin{tabularx}{\textwidth}{|X|}
		  \hline
		  \parbox{\dimexpr\textwidth-2\tabcolsep\relax}{
			  \strut
			  \textbf{Couleur globale}
			  \begin{center}
				  \renewcommand{\arraystretch}{1.3}
				  \begin{tabular}{| *{12}{C{0.8cm}|} }
					  \hline
					  R1 & R2 & R3 & O1 & O2 & O3 & V1 & V2 & V3 & B1 & B2 & B3 \\
					  \hline
					  % \rule{0pt}{6ex}\answerrow \\ % Use the answer row constructed above
					  % \hline
				  \end{tabular}
			  \end{center}
		  } \\
		  \hline
		  \parbox{\dimexpr\textwidth-2\tabcolsep\relax}{
			  \strut
			  \textbf{Couleur par question}
			  % \vspace{1.2cm}
			  \begin{center}
				  \renewcommand{\arraystretch}{1.5}
				  \begin{tabular}{| *{\value{qcntvalue}}{C{1.5cm}|} }
					  \hline
					  \tableheader \\ % Use the header constructed above
					  \hline
					  % \rule{0pt}{6ex}\answerrow \\ % Use the answer row constructed above
					  % \hline
				  \end{tabular}
			  \end{center}
		  } \\
		  \hline
		  \parbox{\dimexpr\textwidth-2\tabcolsep\relax}{
			  \strut
			  \textbf{Remarques}
		  \\\versions{}{}{}{Version EABS.}
			  \vspace{1.2cm}
			  \strut
		  } \\
		  \hline
	  \end{tabularx}
  \end{spacing}
 }

\subsection*{Objectifs}
% Targeted competences and table for grading can be set here. For example:
{ \small
	\begin{spacing}{1.5}
		\begin{tabularx}{\textwidth}{|X| *{2}{c|}}
			\hline
			% \textbf{Je suis capable de...}                                                                      & \textbf{1} & \textbf{2} & \textbf{3} & \textbf{4} & \textbf{5} \\
			\textbf{Je suis capable de...}                                                                      &           &           \\
			\hline
			Effectuer des divisions Euclidienne par calcul écrit.                                               & \ding{51} & \ding{55} \\
			\hline
			Formuler et utiliser dans différents contextes la relation fondamentale de la division Euclidienne. & \ding{51} & \ding{55} \\
			\hline
			Reconnaitre des nombres premiers entre eux.                                                         & \ding{51} & \ding{55} \\
			\hline
			Déterminer l’ensemble des diviseurs et des multiples d’un nombre.                                   & \ding{51} & \ding{55} \\
			\hline
			Déterminer les diviseurs communs et le PGCD de deux nombres ou plus.                                & \ding{51} & \ding{55} \\
			\hline
			Déterminer les multiples communs et le PPCM de deux nombres ou plus.                                & \ding{51} & \ding{55} \\
			\hline
			Résoudre des problèmes relatifs au PGCD ou PPCM.                                                    & \ding{51} & \ding{55} \\
			\hline
		\end{tabularx}

	\end{spacing}
}

